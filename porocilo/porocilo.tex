\documentclass[11pt,a4paper]{article}

\usepackage[slovene]{babel}
\usepackage[utf8x]{inputenc}
\usepackage{graphicx}

\pagestyle{plain}

\begin{document}
\title{Poročilo pri predmetu \\
Analiza podatkov s programom R}
\author{Neja Pletikos}
\maketitle

\section{Izbira teme}

V prvi fazi sem si izbrala temo in sicer prometne nesreče po regijah v Sloveniji. Primerjala bom število prometnih nesreč po vrstah poškodbe (s smrtnim izidom, telesno poškodbo ali zgolj materialna škoda) in po poškodbah udeležencev v prometnih nesrečah (smrtne žrtve, huda/lažja/brez poškodbe). Poskušala bom ugotoviti kako se po regijah število prometnih nesreč razlikuje, ali katera posebej izstopa s številom smrtnih žrtev, zgolj po številu prometnih nesreč itd.

\section{Obdelava, uvoz in čiščenje podatkov}

V drugi fazi sem uvozila tabelo s podatki v CSV obliki. Pri tem sem si pomagala z vajami ki smo jih rešili na faksu in pa z vzorcem ki je bil priložen. Naredila sem še dva grafa, eden prikazuje število prometnih nesreč v letih 2011-2013, drugi pa število udeležencev v prometnih nesrečah v istem časovnem obdobju.

Graf števila prometnih nesreč v letih 2011-2013:

\includegraphics[width=\textwidth]{../slike/skupajnesrece.pdf}

Graf števila udeležencev:

\includegraphics[width=\textwidth]{../slike/steviloudelezencev.pdf}

\section{Analiza in vizualizacija podatkov}

V tretji fazi sem uvozila zemljevid Slovenije z vrisanimi regijami, na njem so napisana še imena posameznih regij. Na njemu je prikazano število nesreč v vsaki regiji v letu 2011. Iz zemljevida je razvidno, da je bilo največ nesreč v Osrednjeslovenski in Podravski regiji.

\makebox[\textwidth][c]{
\includegraphics[width=1.2\textwidth]{../slike/slovenija1.pdf}
}

\section{Napredna analiza podatkov}



\end{document}
